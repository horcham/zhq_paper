\section{前言}
大量数据代表了价值。数据背后通常隐含着客观规律,如果数据量足够大的话,其规律是可以被认知和学习的,其催生了机器学习的研究方向,研究如何用数据进行建模与变现。然而,由于数据量极大,而且所涉及的算法会很复杂,通常不可能进行人为的计算,即使是用计算机进行计算,也对计算机的处理速度,内存,储存空间提出了一定的要求。另一方面,如若要进行机器学习,除了计算机硬件的要求之外,还需要软件与算法的支持,其中,算法是机器学习的核心。历史发展来看,计算机硬件,用于机器学习的软件与算法的发展是相辅相成的。

在20世纪40年代,人们开始研究人工智能,由于生物学的发展,人们模仿人类的神经元运作而提出了神经网络的原型:M-P神经元模型,并提出了激活函数的概念。在20世纪50年代到60年代,感知器算法、梯度下降法、最小二乘法等求解算法面世,而且提出了感知器,并开始应用在文字、语音、信号等领域。在20世纪60年代到70年代,神经网络算法因感知器的缺陷而衰落。在70年代到80年代,神经网络的种类变得丰富起来,涌现出BP神经网络,RBF神经网络等各种网络,并提出了深度学习的概念与卷积神经网络(CNN)和循环神经网络(RNN)的结构。90年代后,一些有别于神经网络的算法面世,如SVM,决策树,boosting与随机森林等方法,从不同的角度对机器学习算法进行丰富。在2006年,Hinton提出了解决深度学习中梯度消失问题的解决方法之后,深度学习开始爆发。2012年,ReLU激活函数的提出,进一步抑制了梯度消失的问题,并且深度学习在语音和图像方面开始有惊人的表现。2012年,在ImageNet图像识别比赛上,AlexNet通过构建一定深度的CNN夺得冠军,其性能彻底击败了SVM。需注意的是,AlexNet首次使用了ReLU激活函数,Dropout防止过拟合方法,以及GPU加速。之后,在AlexNet的结构上做优化,又提出了其他更强大的模型,如VGGNet,Inception系列,ResNet等。强化学习和迁移学习的提出,进一步增强了模型的性能。

本论文基于kaggle(全球数据科学平台)的花苗分类竞赛(Plant Seedlings Classification\footnote{\url{https://www.kaggle.com/c/plant-seedlings-classification}})中的数据集,探究传统机器学习算法(SVM,决策树,随机森林与boosting等)、深度学习算法(AlexNet,VGGNet,InceptionV3)的原理与性能,并对其尝试做优化与结合(如AlexNet+SVM等)。
