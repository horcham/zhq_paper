\section{RCNN}
\subsection{神经网络}
\subsection{卷积神经网络}
\subsubsection{AlexNet}
\subsubsection{VGGNet}
\subsection{Inception}
\subsection{ResNet}

\subsection{迁移学习}
\subsection{交并比}
交并比(Intersection-over-Union,Iou)是用于衡量候选框(candidate bound)与标记框(ground truth bound)相似程度的指标,其是两者的交叠率。假设候选框区域为$C$,标记框区域为$G$,则定义交并比$IoU$为
\begin{eqnarray}
IoU=\frac{C\cap G}{C\cup G}
\end{eqnarray}
%如图所示
若$IoU$越接近与1,说明重叠程度越大,效果越好。
\subsection{非极大值抑制}
非极大值抑制算法(Non-maximum suppression,NMS)本质为搜索局部的极大值,并抑制附近非极大值的元素。

若给定一个$n$个元素的一维数组$A$,该数组的顺序已定义了该数组的序。并定义领域$\epsilon$,对于某个元素$A[i]$,若$A[i]>A[j],j\in[i-\epsilon,i]$且$A[i]<A[j],j\in[i,i+\epsilon]$,则$A[i]$为极大值,否则,跳出$\epsilon$范围并重复上述操作,直到数组遍历完毕。其算法伪代码如下,其中,设$\epsilon=2$
\begin{lstlisting}[language=python]
Input: A
Output: MaximumSet
i=2
while i<=n-1
    if A[i]>A[i-1]
    	if A[i]>A[i+1]
            MaximumSet = MaximumSet `$\cup$` A[i]
    else
    	i=i+1
    	while i<=n-1 and A[i]<=A[i+1]
    		i=i+1
    	if i<=n-1
    		MaximumSet = MaximumSet `$\cup$` A[i]
    i=i+2
\end{lstlisting}

在物体检测中,由于初始化的候选边框数量很大,对于要检测的物体,其对应的边框很多,且边框之间的交叉重复特别严重,因此考虑用非极大值抑制来找到最佳的边框。

\subsection{选择性搜索}

\subsection{候选区域变换}
由于使用选择性搜索算法所产生的候选区域是长方形的,但其所包含的元素不定。又因为需要使用卷积神经网络对这些候选区域进行特征提取与降维,这就意味着需要将不定大小的候选区域变换为一个长宽(假设是$a\times b$)固定的区域。常见的方法如下
\paragraph{tightest square with context}该方法考虑的,首先是采用各向同性的方法,将候选区域扩展为$\max\{a,b\}\times\max\{a,b\}$的正方形区域。对于候选区域扩展后无法涉及的区域,用原来的图像对应的像素进行填补,之后再对该正方形进行裁剪,裁剪为$a\times b$。该想法可理解为扩展后加入背景。
\paragraph{tightest square without context}该方法是 tightest square with context的变体,其在候选区域扩展后无法涉及的区域的处理方法有所不同。其考虑的是对这部分区域不做处理。
\paragraph{warp}该方法采用的是各项异性的方法,直接把原来的图像,从长宽方向使用各自的比例进行放缩,直接放缩为$a\times b$ 。
