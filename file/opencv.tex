\section{数据预处理}
\subsection{掩模构建}
对于花苗图像,我们可以看到,花苗的背景通常为黄土、砂砾或塑料箱等,而绿色的花苗则显得非常好辨认。而且我们面对的花苗是绿色的,因而考虑设置一个hsv范围,将绿色的部分从图像中剥离出来。于是我们首先将花苗图像进行颜色空间的转换,从rgb颜色空间转化为hsv颜色空间,之后设定hsv颜色空间为

!!!hsv颜色空间

,从而筛选去绿色部分。如图所示

!!!原图,过滤出来的绿色图像,mask

\subsection{形态学去噪}
对于用掩模处理后的花苗二值图像,考虑到在花苗所在盆栽可能会有一些小草,通过掩模处理后会有噪声。因而考虑用形态学方法去噪。

!!!带噪声的图片

对于一个二值图像,比较常用的去噪方法是形态学去噪,而这通常涉及两种形态学转换,分别为腐蚀和膨胀,其涉及的原理较简单。对于腐蚀,先定义一个窗口,窗口将沿着图像滑动,以遍历整个图像。滑动过程中,窗口内所有像素不全为1时,则令窗口中的所有像素等于0;若窗口内所有像素全为1,时,则不做操作。选用一个合适尺寸的窗口,对于腐蚀之后的图片,其白噪声点可以消除,但也会对物体的边缘进行腐蚀。膨胀则与腐蚀相反,区别在于滑动过程中,窗口内元素只要有1,则整个窗口元素都令为1,这样会增大物体的尺寸。通常对于有白噪声的图片,先腐蚀再膨胀可以消除白噪声,但一定程度会导致物体失真。但由于用掩模处理后的图像,其物体十分明显,用形态学方法去噪后失真的可能性不大。因而考虑用形态学方法去噪。

!!!去噪后的图片

\subsection{边框裁剪与尺寸归一化}

